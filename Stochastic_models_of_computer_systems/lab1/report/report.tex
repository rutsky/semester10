% Report.
%
% Copyright (C) 2011  Vladimir Rutsky <altsysrq@gmail.com>
%
% This work is licensed under a Creative Commons Attribution-ShareAlike 3.0 
% Unported License. See <http://creativecommons.org/licenses/by-sa/3.0/> 
% for details.

\documentclass[a4paper,10pt]{article}

% Encoding support.
\usepackage{ucs}
\usepackage[utf8x]{inputenc}
\usepackage[T2A]{fontenc}
\usepackage[russian]{babel}

\usepackage{amsmath, amsthm, amssymb}

% Indenting first paragraph.
\usepackage{indentfirst}

\usepackage{url}
\usepackage[unicode]{hyperref}

%\usepackage[final]{pdfpages}

\usepackage[pdftex]{graphicx}
\usepackage{subfig}

\usepackage{fancyvrb}
\usepackage{color}
\usepackage{texments}

\newcommand{\HRule}{\rule{\linewidth}{0.5mm}}

% Spaces after commas.
\frenchspacing
% Minimal carrying number of characters,
\righthyphenmin=2

% From K.V.Voroncov Latex in samples, 2005.
\textheight=24cm   % text height
\textwidth=16cm    % text width.
\oddsidemargin=0pt % left side indention
\topmargin=-1.5cm  % top side indention.
\parindent=24pt    % paragraph indent
\parskip=0pt       % distance between paragraphs.
\tolerance=2000
%\flushbottom       % page height aligning
%\hoffset=0cm
%\pagestyle{empty}  % without numeration

\newcommand{\myemail}[1]{%
\href{mailto:#1}{\nolinkurl{#1}}}

\newcommand{\myfunc}[1]{%
\textit{#1}}

\begin{document}

% Title page.
% Report title page.
%
% Copyright (C) 2011  Vladimir Rutsky <altsysrq@gmail.com>
%
% This work is licensed under a Creative Commons Attribution-ShareAlike 3.0 
% Unported License. See <http://creativecommons.org/licenses/by-sa/3.0/> 
% for details.

\begin{titlepage} % начало титульной страницы

\begin{center} % включить выравнивание по центру

\large Санкт-Петербургский государственный политехнический университет\\[5.5cm]
% название института, затем отступ 5,5см

\huge Лабораторная работа~\No\,1\\[0.6cm] % название работы, затем отступ 0,6см
\large по~курсу <<Стохастические модели>>\\[1cm]
\large <<Проверка гипотезы о присутствии закодированного сообщения в наборе кодов>>\\[6cm]
% тема работы, затем отступ 6см

\begin{flushright} % выровнять её содержимое по правому краю
\begin{tabular}{l l}
Студент: & Руцкий~В.\,В.\\
Группа: & 5057/2\\
Преподаватель: & Иванков ~А.\,А.
\end{tabular}
\end{flushright} % конец выравнивания по правому краю

\vfill % заполнить всё доступное ниже пространство

{\large Санкт-Петербург 2011}
\end{center} % закончить выравнивание по центру
\thispagestyle{empty} % не нумеровать страницу
\end{titlepage} % конец титульной страницы

%\tableofcontents
%\pagebreak

% Content

\section{Постановка задачи}
Дано сообщение~--- упорядоченный набор символов~--- 
$M = (s_0, \ldots, s_n), \quad s_i \in \Sigma, \quad \Sigma = \{ c_0, \ldots, c_k \}$%
~--- алфавит.
Необходимо проверить гипотезу о том, что в сообщении закодирован текст $M_0$
на английском языке, при условии, что кодирование было произведено 
переобозначением с помощью какой-то биекции 
$f \colon \Sigma \rightarrow \Sigma$ 
исходных символов новыми: 
$M = (f(M_0[i]), \ldots, f(M_0[n]))$.

\section{Решение}
\paragraph{Формализация гипотезы}
Рассмотрим сообщение $M_0$ как $n$ наблюдений случайной величины $S$, 
принимающей значения из $\Sigma$.
Для английского языка определённой стилистики $S$ подчиняется некоторому
закону распределения, который можно считать известным.

\paragraph{Проверка гипотезы}
Рассмотрим все биекции на $\Sigma$: 
$F = \{ f | f \colon \Sigma \rightarrow \Sigma \}$.
Предположим, что $f_j \in F$~--- биекция, которой было закодировано сообщение 
$M_0$.
Гипотеза $H_0$ которую необходимо проверить состоит в том, 
что случайная величина $S'$, для которой наблюдается выборка 
$M' = (f_j^{-1}(M[0]), \ldots, f_j^{-1}(M[n]))$,
подчиняется такому же закону распределения, что и $S$.

Проверим гипотезу с помощью критерия согласия Пирсона:
$$ \chi^2 = \sum_{i=1}^{k}\frac{(v_i - n p_i)^2}{n p_i}, $$
$p_i$~--- теоретическая вероятность наблюдения $c_i$ в $S$,
$v_i$~--- число наблюдений $c_i$ в выборке $M'$.
Выборочная характеристика $\chi^2$ при $n \rightarrow \infty$ имеет 
$\chi^2$-распределение с $k - 1$ степенями свободы.

Теперь для выбранного уровня значимости $\alpha$, сравним полученную величину
$\chi^2$ с критической величиной $\chi^2_{cr}(k, \alpha)$ (табличная величина): 
если $\chi^2 < \chi^2_{cr}(k, \alpha)$, то гипотеза принимается 
с уровнем значимости $\alpha$.

Если хотя бы для одного $f_i$ принимается $H_0$, то в сообщении закодирован 
текст на английском языке, иначе~--- нет.

%В ходе решения были использованы из \cite{wiki:chi-square}, 
%\cite{wiki:pearson},
%\cite{wiki:chi-square-qu},
%\cite{wiki:chi-square2},
%\cite{chernova2006matstat}.
%\cite{wiki:chi-square4}.

%\appendix
%\section{Исходный код}
%\label{appendix:sources}

%\usestyle{default}

%\subsection{Title}
%\label{appendix:sources:sources-name}
%\includecode[python -O linenos=1]{data/source.py}

\pagebreak

\bibliographystyle{unsrt}
\bibliography{references}

\end{document}
