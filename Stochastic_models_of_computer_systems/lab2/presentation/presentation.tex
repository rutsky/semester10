% Presentation for second lab.
%
% Copyright (C) 2011  Vladimir Rutsky <altsysrq@gmail.com>
%
% This work is licensed under a Creative Commons Attribution-ShareAlike 3.0
% Unported License. See <http://creativecommons.org/licenses/by-sa/3.0/>
% for details.

\documentclass[utf8]{beamer}
\usepackage[russian]{babel}
\usepackage{commath}
%\usetheme{default}
\usetheme{Madrid}
%\usetheme{Warsaw}
%\usetheme{Darmstadt}

% Убирает меню навигации по слайдам.
\setbeamertemplate{navigation symbols}{}

% Футер: номер слайда / общее число слайдов.
%\setbeamertemplate{footline}
%{\centerline{\insertframenumber/\inserttotalframenumber}}

% Футер: Автор, номер слайда
%\setbeamertemplate{footline}{\hspace*{.5cm}\scriptsize{\insertauthor
%\hspace*{50pt} \hfill\insertframenumber\hspace*{.5cm}}}

\defbeamertemplate*{footline}{my split theme}
{%
  \leavevmode%
  \hbox{\begin{beamercolorbox}[wd=.5\paperwidth,ht=2.5ex,dp=1.125ex,leftskip=.3cm plus1fill,rightskip=.3cm]{author in head/foot}%
    \usebeamerfont{author in head/foot}\insertshortauthor
  \end{beamercolorbox}%
  \begin{beamercolorbox}[wd=.4\paperwidth,ht=2.5ex,dp=1.125ex,leftskip=.3cm,rightskip=.3cm]{title in head/foot}%
    \usebeamerfont{title in head/foot}\insertshorttitle
  \end{beamercolorbox}%
  \begin{beamercolorbox}[wd=.1\paperwidth,ht=2.5ex,dp=1.125ex,leftskip=.4cm,rightskip=.3cm plus1fil]{title in head/foot}%
    \usebeamerfont{title in head/foot}\insertframenumber/\inserttotalframenumber
  \end{beamercolorbox}}%
  \vskip0pt%
}

\title[Оценка параметров случайного процесса]{Оценка параметров модели случайного процесса нагрузки сервера, обрабатывающего заявки}
\author{Владимир Руцкий, 5057/12}
\institute[СПбГПУ]{Санкт-Петербургский государственный политехнический университет}
\date{31 мая 2011}

% Показывает план перед каждой сабсекцией.
%\AtBeginSubsection[]
%{
%  \begin{frame}<beamer>
%  \frametitle{План презентации}
%  \tableofcontents[currentsection,currentsubsection]
%  \end{frame}
%}

\begin{document}

\begin{frame}
\titlepage
\end{frame}


\begin{frame}
\frametitle{План презентации}
\tableofcontents
\end{frame}


\section{Постановка задачи}
\begin{frame}{Постановка задачи}
\begin{block}{Сервер}
  \begin{itemize}
    \item Сервер обрабатывает приходящие заявки
    \item Для обработки заявки используются ресурсы сервера
      \begin{itemize}
        \item Количество используемых ресурсов сервера~--- загрузка сервера ~--- скалярная величина, например процессорное время
      \end{itemize}
  \end{itemize}
\end{block}

\begin{block}{Задача}
  Дан лог загрузки сервера. 
  Необходимо:
  \begin{enumerate}
    \item идентифицировать моменты прихода заявок,
    \item оценить:
      \begin{itemize}
        \item интенсивность прихода заявок,
        \item загрузку сервера в фоновом режиме,
        \item использование ресурсов сервера для обработки одной заявки
      \end{itemize}
  \end{enumerate}
\end{block}
\end{frame}


\begin{frame}{Математическая модель (1)}
Загрузка сервера~--- случайный процесс $X(t)\colon \mathbb{R} \rightarrow \mathbb{R}$
\begin{block}{Фоновая загрузка сервера}
  Сумма постоянной загрузки и винеровского процесса (шум): 
  $$B(t) = m + \sigma \mathcal{W}(t)$$
\end{block}
\begin{block}{Загрузка сервера при обработке заявки}
  Заявка, пришедшая в $t=t_c$, увеличивает загрузку сервера на:
  $$K_{t_c}(t) = \mathcal{N}(m_c, \sigma_c^2) \cdot \mathrm{I}(t - t_c) \cdot 
    e^{-\lambda_c(t - t_c)}$$
  {\footnotesize $\mathrm{I}(x) = \left\{
    \begin{array}{rl}
      0, & x < 0 \\
      1, & x \geqslant 0
    \end{array}\right.$~--- Функция Хевисайда}
\end{block}
\end{frame}


\begin{frame}{Математическая модель (2)}
$T_c = \{ t_{c_1}, \ldots, t_{c_R} \}$~--- 
множество моментов времени, когда поступили заявки 
(всего $R$ заявок за время наблюдения за сервером)
\begin{block}{Интенсивность поступления заявок}
  Интервал времени между поступлением двух последовательных заявок распределён экспоненциально:
  $$(t_{c_i} - t_{c_{i-1}}) \sim \mathrm{Exp}(\lambda)$$
\end{block}
\begin{block}{Общая загрузка сервера}
  $$X(t) = B(t) + \sum\limits_{t_c \in T_c}K_{t_c}(t)$$
\end{block}
\begin{block}{Перейдём к дискретному случайному процессу}
  $$X(t_i), \quad t_i = t_0 + i \cdot \Delta t \quad 
    (t_0 \mathrm{\text{ и }} \Delta t \mathrm{\text{ даны}})$$
\end{block}
\end{frame}


\begin{frame}{Входные данные}
\begin{block}{Лог загрузки сервера}
  Траектория $X(t_i)\colon \{ x_i \ \vert\ i = 1, \ldots, N \}$
\end{block}
Считаем, что $\Delta t$ достаточно мало и $t_{c_j} = t_0 + i \cdot \Delta t$~---
заявки пришли в некоторые наблюдаемые моменты времени $t_i.$

Без ограничения общности будем рассматривать случай $t_0 = 0$
\end{frame}


\begin{frame}{Задача}
\begin{block}{Задача}
  По траектории $X(t_i)$ 
  \begin{enumerate}
    \item идентифицировать моменты поступления заявок,
    \item оценить параметры модели:
      \begin{itemize}
        \item $m, \sigma$~--- параметры фоновой загрузки,
        \item $\lambda$~--- интенсивность поступления заявок,
        \item $m_c, \sigma_c, \lambda_c$~--- параметры загрузки сервера при обработке заявок
      \end{itemize}
  \end{enumerate}
\end{block}
\end{frame}


\section[$\lambda_c \approx 0$]{Решение в случае бесконечного времени обработки заявки}
\begin{frame}{Случай бесконечного времени обработки заявки}
\begin{block}{$\lambda_c \approx 0$}
  При поступлении заявки в момент времени $t_c$ загрузка сервера увеличивается на 
  $\Delta x \sim \mathcal{N}(m_c, \sigma_c)$

  $$K_{t_c}(t) = \mathcal{N}(m_c, \sigma_c^2) \cdot \mathrm{I}(t - t_c)$$
\end{block}
\end{frame}


\begin{frame}{Разностный аналог производной}
\begin{block}{$\dif X(t)$}
  Рассмотрим ненормированный разностный аналог производной:
  $$\dif X(t) = X(t) - X(t - \Delta t)$$
\end{block}
\begin{block}{$\dif X(t)$ в момент времени поступления заявки $t_c$}
  $$\dif X(t_c) = \mathcal{N}(m_c, \sigma^2 \Delta t + \sigma_c^2)$$
  (при условии, что в момент времени $(t_c - \Delta t)$ заявки не было)
\end{block}
\begin{block}{$\dif X(t)$ в момент времени отсутствия заявок $t$}
  $$\dif X(t) = \mathcal{N}(0, \sigma^2 \Delta t)$$
  (при условии, что в момент времени $(t - \Delta t)$ заявки не было)
\end{block}
\end{frame}


\subsection{Итеративный метод}
\begin{frame}{Итеративный метод идентификации поступления заявок}
Предположим, что в отрезке времени $\sbr{t_k, t_{k+n}}$ не пришло ни одной заявки.

Тогда $\dif x_k, \dots, \dif x_{k+n}$~--- наблюдения 
$\mathcal{N}(0, \sigma^2 \Delta t) = \dif X(t).$

Оценим $\sigma^2$ по $\sbr{x_k, x_{k+n}}$ (ММП):
$$\Delta t \cdot \widehat{\sigma}^2 = 
    \frac{1}{n-1}
        \sum\limits_{i=1}^n ((x_{k+i} - x_{k+i-1}) - 0)^2$$

\begin{block}{Гипотеза $H_0$}
  В отрезке времени $\sbr{t_{k+n}, t_{k+n} + \Delta t}$ не поступило ни 
  одной заявки
\end{block}
\begin{block}{Критерий принятия $H_0$ с уровнем значимости $\alpha$}
Разность значений наблюдений $(x_{k+n+1}-x_{k+n})$
лежит в $\del{1 - \alpha}$ квантиле
нормального распределения $\mathcal{N}(0, \widehat{\sigma}^2 \Delta t)$:
$$
H_0 \  \mathrm{\text{принимается}} \iff
        (x_{k+n+1}-x_{k+n}) < 
	    \mathcal{N}_{1 - \alpha}
$$
\end{block}
\end{frame}


\begin{frame}

\end{frame}


\subsection[EM-алгоритм]{Оценивание EM-алгоритмом}
\begin{frame}{EM-алгоритм}
\end{frame}

\section[$\lambda_c \gg 0$]{Решение в случае конечного времени обработки заявки}
\subsection[МНК]{Метод наименьших квадратов}
\begin{frame}{Метод наименьших квадратов}
\end{frame}

\section{Результаты работы}
\begin{frame}{Результаты работы (1)}
\end{frame}

\begin{frame}{Выводы}
\end{frame}

%\begin{frame}{Список литературы}
%  \begin{thebibliography}{10}
%    \bibitem{ivchmed2010matstat}[Ивченко, 2010] Г.\,И.~Ивченко, Медведев~Ю.\,И.
%    \emph{{Введение в математическую статистику}}
%    %\newblock A problem we should try to solve before the ISPN ’43 deadline,
%    %\newblock \emph{Letter to Leonhard Euler}, 1742.
%  \end{thebibliography}
%\end{frame}


\end{document}

% vim: set ts=2 sw=2 et:
