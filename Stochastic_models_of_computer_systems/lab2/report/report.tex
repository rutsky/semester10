% Report.
%
% Copyright (C) 2011  Vladimir Rutsky <altsysrq@gmail.com>
%
% This work is licensed under a Creative Commons Attribution-ShareAlike 3.0 
% Unported License. See <http://creativecommons.org/licenses/by-sa/3.0/> 
% for details.

\documentclass[a4paper,10pt]{article}

% Encoding support.
\usepackage{ucs}
\usepackage[utf8x]{inputenc}
\usepackage[T2A]{fontenc}
\usepackage[russian]{babel}

\usepackage{amsmath, amsthm, amssymb}

% Indenting first paragraph.
\usepackage{indentfirst}

\usepackage{url}
\usepackage[unicode]{hyperref}

%\usepackage[final]{pdfpages}

\usepackage[pdftex]{graphicx}
\usepackage{subfig}

\usepackage{fancyvrb}
\usepackage{color}
\usepackage{texments}

\newcommand{\HRule}{\rule{\linewidth}{0.5mm}}

% Spaces after commas.
\frenchspacing
% Minimal carrying number of characters,
\righthyphenmin=2

% From K.V.Voroncov Latex in samples, 2005.
\textheight=24cm   % text height
\textwidth=16cm    % text width.
\oddsidemargin=0pt % left side indention
\topmargin=-1.5cm  % top side indention.
\parindent=24pt    % paragraph indent
\parskip=0pt       % distance between paragraphs.
\tolerance=2000
%\flushbottom       % page height aligning
%\hoffset=0cm
%\pagestyle{empty}  % without numeration

\newcommand{\myemail}[1]{%
\href{mailto:#1}{\nolinkurl{#1}}}

\newcommand{\myfunc}[1]{%
\textit{#1}}

\begin{document}

% Title page.
\input{./title.tex}
%\tableofcontents
%\pagebreak

% Content

\section{Постановка задачи}
Данной работе производится анализ лога загруженности процессора сервера
при поступающих заявках на обработку информации.
% TODO: Лучше рассматривать измерение абстрактных величин~--- загрузка 
% ресурсов сервера.

В отсутствие заявок величина загруженности процессора представляет собой 
сумму некоторой постоянной величины загрузки $m$ и случайных отклонений:
$$B(t) = m + \sigma \mathcal{W}(t),$$
где $\mathcal{W}(t)$~--- это винеровский процесс.

Интенсивность поступления заявок подчиняются закону
распределения Пуассона $\mathcal{P}(\lambda)$.
% TODO: Лучше было бы рассматривать $\lambda = \lambda(t)$

При поступлении одной заявки нагрузка на процессор мгновенно возрастает,
а затем экспоненциально снижается до прежнего уровня.
Увеличение загрузки процессора от одной заявки, 
поступившей в момент времени $t_c$ выражается следующим образом:
$$K_{t_c}(t) = \mathcal{N}(m_c, \sigma_c) \cdot \mathrm{I}(t > t_c) \cdot 
    e^{-\lambda_c(t - t_c)}.$$
% Экспоненциальное падение можно обосновать с помощью ТМО.
% d K(t) = -\lambda K'(t) dt

В логе загруженности процессора наблюдается общая загрузка процессора:
$$X(t) = B(t) + \sum\limits_{t_c \in T_c}K_{t_c}(t),$$
где $T_c$~--- это моменты времени поступления заявок.

Необходимо по дискретным наблюдениям $X(t_i)$
\begin{enumerate}
 \item оценить моменты времени поступления заявок $T_c$,
 \item оценить параметры модели $m$, $\sigma$, $\lambda$, $m_c$, 
 $\sigma_c$, $\lambda_c$.
\end{enumerate}

Для упрощения решения $\lambda_c$ принимается равным величине близкой к нулю,
т.\,е.~каждая пришедшая заявка увеличивает загрузку процессора на некоторую 
фиксированную величину.

\cite{ivchmed2010matstat}

\section{Решение}
\subsection{Оценка моментов времени поступления заявок}
Предположим, что в интервале $(t_i, t_{i+n})$ не пришло ни одной заявки.
Тогда наблюдения $X(t_i),\ldots,X(t_{i+n})$ представляют собой наблюдения
$B(t)$.
Оценим по этим наблюдениям параметры $B(t)$.

При условии, что интервал $(t_i, t_{i+n})$ небольшой, 
$B(t)$ можно считать нормально распределённой случайной величиной:
$$B(t) = \mathcal{N}(m_{t_i}, \sigma_{t_i}^2).$$

Построим точечные оценки $m_{t_i}$ и $\sigma_{t_i}$ 
методом максимального правдоподобия:
$$m_{t_i} = \frac{1}{n}\sum\limits_1^n X(t_i),$$
$$\sigma_{t_i}^2 = \frac{1}{n - 1}\sum\limits_1^n (X(t_i) - m_{t_i})^2.$$

\section{Результаты работы}

%\appendix
%\section{Исходный код}
%\label{appendix:sources}

%\usestyle{default}

%\subsection{Title}
%\label{appendix:sources:sources-name}
%\includecode[python -O linenos=1]{data/source.py}

\pagebreak

\bibliographystyle{unsrt}
\bibliography{references}

\end{document}
