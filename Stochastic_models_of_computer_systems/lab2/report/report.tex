% Report.
%
% Copyright (C) 2011  Vladimir Rutsky <altsysrq@gmail.com>
%
% This work is licensed under a Creative Commons Attribution-ShareAlike 3.0 
% Unported License. See <http://creativecommons.org/licenses/by-sa/3.0/> 
% for details.

\documentclass[a4paper,10pt]{article}

% Encoding support.
\usepackage{ucs}
\usepackage[utf8x]{inputenc}
\usepackage[T2A]{fontenc}
\usepackage[russian]{babel}

\usepackage{amsmath, amsthm, amssymb}
\usepackage{commath}

% Indenting first paragraph.
\usepackage{indentfirst}

\usepackage{url}
\usepackage[unicode]{hyperref}

%\usepackage[final]{pdfpages}

\usepackage[pdftex]{graphicx}
\usepackage{subfig}

\usepackage{fancyvrb}
\usepackage{color}
\usepackage{texments}

\newcommand{\HRule}{\rule{\linewidth}{0.5mm}}

% Spaces after commas.
\frenchspacing
% Minimal carrying number of characters,
\righthyphenmin=2

% From K.V.Voroncov Latex in samples, 2005.
\textheight=24cm   % text height
\textwidth=16cm    % text width.
\oddsidemargin=0pt % left side indention
\topmargin=-1.5cm  % top side indention.
\parindent=24pt    % paragraph indent
\parskip=0pt       % distance between paragraphs.
\tolerance=2000
%\flushbottom       % page height aligning
%\hoffset=0cm
%\pagestyle{empty}  % without numeration

\newcommand{\myemail}[1]{%
\href{mailto:#1}{\nolinkurl{#1}}}

\newcommand{\myfunc}[1]{%
\textit{#1}}

\DeclareMathOperator*{\argmax}{\arg\!\max}

\begin{document}

% Title page.
% Report title page.
%
% Copyright (C) 2011  Vladimir Rutsky <altsysrq@gmail.com>
%
% This work is licensed under a Creative Commons Attribution-ShareAlike 3.0 
% Unported License. See <http://creativecommons.org/licenses/by-sa/3.0/> 
% for details.

\begin{titlepage} % начало титульной страницы

\begin{center} % включить выравнивание по центру

\large Санкт-Петербургский государственный политехнический университет\\[5.5cm]
% название института, затем отступ 5,5см

\huge Лабораторная работа~\No\,1\\[0.6cm] % название работы, затем отступ 0,6см
\large по~курсу <<Стохастические модели>>\\[1cm]
\large <<Проверка гипотезы о присутствии закодированного сообщения в наборе кодов>>\\[6cm]
% тема работы, затем отступ 6см

\begin{flushright} % выровнять её содержимое по правому краю
\begin{tabular}{l l}
Студент: & Руцкий~В.\,В.\\
Группа: & 5057/2\\
Преподаватель: & Иванков ~А.\,А.
\end{tabular}
\end{flushright} % конец выравнивания по правому краю

\vfill % заполнить всё доступное ниже пространство

{\large Санкт-Петербург 2011}
\end{center} % закончить выравнивание по центру
\thispagestyle{empty} % не нумеровать страницу
\end{titlepage} % конец титульной страницы

%\tableofcontents
%\pagebreak

% Content

\section{Постановка задачи}
Данной работе производится анализ лога загруженности процессора сервера
при поступающих заявках на обработку информации.
% TODO: Лучше рассматривать измерение абстрактных величин~--- загрузка 
% ресурсов сервера.

В отсутствие заявок величина загруженности процессора представляет собой 
сумму некоторой постоянной величины загрузки $m$ и случайных отклонений:
$$B(t) = m + \sigma \mathcal{W}(t),$$
где $\mathcal{W}(t)$~--- это винеровский процесс.

Интенсивность поступления заявок подчиняются закону
распределения Пуассона $\mathcal{P}(\lambda)$.
% TODO: Лучше было бы рассматривать $\lambda = \lambda(t)$

При поступлении одной заявки нагрузка на процессор мгновенно возрастает,
а затем экспоненциально снижается до прежнего уровня.
Увеличение загрузки процессора от одной заявки, 
поступившей в момент времени $t_c$ выражается следующим образом:
$$K_{t_c}(t) = \mathcal{N}(m_c, \sigma_c^2) \cdot \mathrm{I}(t - t_c) \cdot 
    e^{-\lambda_c(t - t_c)},$$
где $\mathrm{I}(x)$~--- фунцкия Хевисайда.%
\footnote{%
Функция Хевисайда: $\mathrm{I}(x) = \left\{
  \begin{array}{rl}
    0, & x < 0 \\
    1, & x \geqslant 0
  \end{array}\right.$.
}
% Экспоненциальное падение можно обосновать с помощью ТМО.
% d K(t) = -\lambda K'(t) dt

В логе загруженности процессора наблюдается общая загрузка процессора:
$$X(t) = B(t) + \sum\limits_{t_c \in T_c}K_{t_c}(t),$$
где $T_c$~--- это моменты времени поступления заявок.

Необходимо по дискретным наблюдениям $x_i$ случайного процесса $X(t)$ 
в моменты времени $t_i, \quad i=1,\ldots,N$
\begin{enumerate}
 \item оценить моменты времени поступления заявок $T_c$,
 \item оценить параметры модели $m$, $\sigma$, $\lambda$, $m_c$, 
 $\sigma_c^2$, $\lambda_c$.
\end{enumerate}
Наблюдения производятся через равные промежутки времени 
$\Delta t = t_{i+1} - t_i.$

Для упрощения решения $\lambda_c$ принимается равным величине близкой к нулю,
т.\,е.~каждая пришедшая заявка увеличивает загрузку процессора на некоторую 
фиксированную величину.

\section{Решение}
\subsection{Идентификация моментов времени поступления заявок}%
\label{find-requests}
Предположим, что в отрезке времени $[t_k, t_{k+n+1}]$ 
не пришло ни одной заявки.
Тогда наблюдения $x_k,\ldots,x_{k+n+1}$ представляют собой наблюдения
$B(t)$.
Оценим по этим наблюдениям параметры $B(t)$.

Рассмотрим разности соседних наблюдений~--- они представляют собой наблюдения 
нормально распределённой случайной величины:
$$B(t_{i+1}) - B(t_i) = 
    \sigma \mathcal{W}(t_{i+1}) - \sigma \mathcal{W}(t_i) = 
    \sigma \mathcal{N}(0, \Delta t) = 
    \mathcal{N}(0, \sigma^2 \Delta t).$$

Построим точечную оценку $\widehat{\sigma}^2$ 
методом максимального правдоподобия:%
\footnote{См.~\S\,3.5 пункт~1 в~\cite{ivchmed2010matstat}.}
%
$$\widehat{\sigma}^2 = 
    \frac{1}{\Delta t}\cdot\frac{1}{n-1}
        \sum\limits_{i=1}^n ((x_{k+i+1} - x_{k+i}) - 0)^2.$$

Обозначим гипотезу о том, что в промежутке времени 
$[t_{k+n+1}, t_{k+n+2}]$ не пришло ни одной заявки, 
как $H_0$.
Тогда 
$$\del{X(t_{k+n+2})-X(t_{k+n+1}) \vert H_0} = 
    \mathcal{N}(0, \sigma^2 \Delta t).$$

В качестве критерия принятия гипотезы $H_0$ с уровнем значимости 
$\alpha$ 
возьмём условие, что разность значений наблюдений $(x_{k+n+2}-x_{k+n+1})$
лежит между $\frac{\alpha}{2}$ и $\del{1 - \frac{\alpha}{2}}$ квантилями
нормального распределения $\mathcal{N}(0, \sigma^2 \Delta t)$,
обозначенные соответственно $\mathcal{N}_{\frac{\alpha}{2}}$ и 
$\mathcal{N}_{1 - \frac{\alpha}{2}}$:
$$
H_0 \  \mathrm{\text{принимается}} \iff
    \mathcal{N}_{\frac{\alpha}{2}} < 
        (x_{k+n+2}-x_{k+n+1}) < 
	    \mathcal{N}_{1 - \frac{\alpha}{2}}.
$$

Алгоритм нахождения моментов времени поступления заявок $T_c$ 
состоит в следующем:
\begin{enumerate}
  \item В предположении, что в первые $n+1$ наблюдений не пришло ни одной 
  заявки, оценим $\widehat{\sigma}$ и построим критерий для принятия $H_0$.
  \item Будем добавлять к первым $n+1$ наблюдениям по одному наблюдению и 
  проверять гипотезу $H_0$.
  Если $H_0$ принимается, то $\widehat{\sigma}$ и критерий для принятия $H_0$ 
  пересчитываются для добавленного наблюдения.
  \item Как только встретиться наблюдение $n+1+l$, для которого гипотеза $H_0$
  отвергается, то \mbox{$t_{n+1+l} \in T_c$}. 
  Все наблюдения до $t_{n+1+l+1}$ отбрасываются и алгоритм начинается с шага 1
  для поиска следующего момента времени прихода заявки.
\end{enumerate}

\subsection{Оценка интенсивности поступления заявок $\lambda$}
Зная время прибытия заявок $T_c$ интенсивность поступления заявок можно оценить
методом максимального правдоподобия:
\footnote{{\url{http://en.wikipedia.org/wiki/Poisson\_distribution\#Maximum\_likelihood}} 
или в общем случае в \S\,3.5 пункт~1 в~\cite{ivchmed2010matstat}.}
$$\widehat{\lambda} = 
    \frac{1}{|T_c|} \sum\limits_{i=0}^{|T_c| - 1} (t_{c_{i+1}} - t_{c_i}).$$

\subsection{Оценка параметров распределения величины нагрузки %
поступающих заявок}
Рассмотрим ненормированный разностный аналог производной 
случайного процесса $X(t)$:
$$\dif X(t) = X(t) - X(t - \Delta t).$$

$\operatorname{d}X(t)$ в момент времени прихода заявки $t_c$ 
выражается следующим образом:
\begin{eqnarray*}
\dif X(t_c) 
  & = & X(t_c) - X(t_c - \Delta t) = 
      B(t_c) + K_{t_c}(t_c) - B(t_c - \Delta t) = \\
  & = & \sigma \mathcal{W}(t_c) - \sigma \mathcal{W}(t_c - \Delta t) + 
      \mathcal{N}(m_c, \sigma_c^2) = \\
  & = & \sigma \mathcal{N}(0, \Delta t) + \mathcal{N}(m_c, \sigma_c^2) =
      \mathcal{N}(m_c, \sigma^2 \Delta t + \sigma_c^2),
\end{eqnarray*}
предполагая, что в момент времени $t_c - \Delta t$ заявки не было.
%Число таких наблюдений равно $|T_c|$.
% NOTE: Здесь можно было бы обойтись простой оценкой параметров нормального
% распределения, но по условию нужно было привлечь что-то из МНК и 
% EM-алгоритма.

Во время отсутствия заявок $\dif X(t)$ выражается как:
$$
\dif X(t) = X(t) - X(t - \Delta t) = 
    B(t) - B(t - \Delta t) =
    \mathcal{N}(0, \sigma^2 \Delta t).
$$

Значит в каждый отдельно взятый момент времени $t$ 
случайная величина $\dif X(t)$ представляет собой смесь двух нормально 
распределённых случайных величин, 
причем параметры случайных величин со временем не меняются.
Оценим их параметры EM-алгоритмом 
(на основе примера из~\cite{wiki:em-algorithm}).

Введём скрытые случайные величины $Z_i, \quad i=1,\ldots,N$, 
принимающие значения 1 или 2, в зависимости от того, 
пришла ли заявка в момент времени $t_i$ или нет соответственно, 
а~$z_i$~--- наблюдения $Z_i$ в момент времени $t_i$.
$$
\begin{array}{rclcll}
\dif X(t_i) \vert (Z_i = 1) &\sim& \mathcal{N}(\mu_1, \sigma_1^2) &=& 
  \mathcal{N}(m_c, \sigma^2 \Delta t + \sigma_c^2), &
  \quad (\mathrm{\text{случай}}~t_i \in T_c), \\
\dif X(t_i) \vert (Z_i = 2) &\sim& \mathcal{N}(\mu_2, \sigma_2^2) &=& 
  \mathcal{N}(0, \sigma^2 \Delta t), &
  \quad (\mathrm{\text{случай}}~t_i \notin T_c).
\end{array}
$$
Пусть $\mathbf{P}(Z_i=1) = \tau_1$ и $\mathbf{P}(Z_i=2) = 
\tau_2 = 1 - \tau_1.$

Введём обозначения: 
$\theta = (\tau_1, \tau_2, \mu_1, \mu_2, \sigma_1^2, \sigma_2^2),$
$\mathbf{x} = (x_1, \ldots, x_N),$
$\mathbf{z} = (z_1, \ldots, z_N).$

Построим функцию правдоподобия:
$$
L(\theta; \mathbf{x}, \mathbf{z}) = 
  \mathbf{P}(\mathbf{x}, \mathbf{z} \vert \theta) = 
  \prod\limits_{i=1}^{N} \sum\limits_{j=1}^2 
    \mathbb{I}(z_i=j) \, \tau_j \, f(x_i,\mu_j,\sigma_j^2),
$$
где $\mathbb{I}(\mathrm{expr})$~--- функция индикатор,%
\footnote{%
Функция индикатор: $\mathbb{I}(\mathrm{expr}) = \left\{
  \begin{array}{rl}
    0, & \mathrm{expr} = \mathrm{False} \\
    1, & \mathrm{expr} = \mathrm{True}
  \end{array}\right.$.
} а $f(x, \mu, \sigma^2)$~--- это функция плотности распределения, 
в данном случае нормального:
$$
  f(x, \mu, \sigma^2) = 
    \frac{1}{\sqrt{2 \pi \sigma^2}} e^{-\frac{(x - \mu)^2}{2 \sigma^2}}.
$$

Перепишем функцию правдоподобия в экспоненциальной форме:
$$
L(\theta; \mathbf{x}, \mathbf{z}) =
  \exp \cbr{ \sum\limits_{i=1}^{N} \sum\limits_{j=1}^2 
    \mathbb{I}(z_i=j) \sbr{
      \log \tau_j - 
      \frac{1}{2}\log(2\pi) -
      \log(\sigma_j) -
      \frac{(x_i - \mu_j)^2}{2 \sigma_j^2}}}.
$$

Пусть имеется начальная оценка параметров $\theta$: $\theta^{(0)}$ 
(использованный способ вычисления $\theta^{(0)}$ будет описан ниже).
Последовательно выполняя E- и M-шаги будем уточнять оценку $\theta^{(k)}$,
пока она не сойдётся к какой-то величине 
$\theta^{(k)} \xrightarrow[k \rightarrow \infty]{} \theta$, 
которую и примем за результат.

\paragraph*{E-шаг}%
Имея текущую оценку параметров $\theta^{(k)}$, вычислим по теореме Байеса 
условную вероятность принадлежности $i$-го наблюдения $j$-му нормальному 
распределению:
$$
T_{j,i}^{(k)} = \mathbf{P}(Z_i = j \vert X(t_i) = x_i; \theta^{(k)}) =
  \frac{\tau_j^{(k)} f(x_i; \mu_j^{(k)}, \sigma_j^{(k)})}
       {\tau_1^{(k)} f(x_i; \mu_1^{(k)}, \sigma_1^{(k)}) + 
       {\tau_2^{(k)} f(x_i; \mu_2^{(k)}, \sigma_2^{(k)})}}.
$$

Построим функцию~--- математическое ожидание логарифма функции правдоподобия:
$$
Q\del{\theta \vert \theta^{(k)}} = 
  \mathbf{E}\sbr{\log L(\theta; \mathbf{x}, \mathbf{z})} =
  \sum\limits_{i=1}^{N} \sum\limits_{j=1}^2
    T_{j,i}^{(k)} 
      \sbr{
        \log \tau_j - 
        \frac{1}{2}\log(2\pi) -
        \log(\sigma_j) -
        \frac{(x_i - \mu_j)^2}{2 \sigma_j^2}
      }.
$$

\paragraph*{M-шаг}%
Теперь найдём параметры $\theta^{(k+1)}$ максимизирующие 
$Q\del{\theta \vert \theta^{(k)}}$:
$$
\theta^{(k+1)} = \argmax\limits_{\theta} Q\del{\theta \vert \theta^{(k)}}.
$$
В соответствии с вычислениями в \cite{wiki:em-algorithm}:
$$
\tau_j^{(k+1)} = \frac{1}{n} \sum\limits_{i=1}^{N} T_{j,i}^{(k)},\quad
\mu_j^{(k+1)} = \frac
  {\sum\limits_{i=1}^{N} T_{j,i}^{(k)} x_i}
  {\sum\limits_{i=1}^{N} T_{j,i}^{(k)}},\quad
\sigma_j^{(k+1)} = \frac
  {\sum\limits_{i=1}^{N} T_{j,i}^{(k)} (x_i - \mu_j^{(k+1)})^2}
  {\sum\limits_{i=1}^{N} T_{j,i}^{(k)}}.
$$

\paragraph*{Вычисление $\theta_0$}
$\tau_j^{(0)}$ вычислим из информации о $T_c$, полученной в пункте 
\ref{find-requests}:
$$
\tau_1^{(0)} = \frac{\envert{T_c}}{N}, \quad
\tau_2^{(0)} = 1 - \frac{\envert{T_c}}{N}.
$$

Для вычисления $\mu_j^{(0)}$ построим полигон частот\footnote{%
Cм.~\S\,2.1 пункт 4 в~\cite{ivchmed2010matstat}.
} $\dif x_i$:
в качестве $\mu_1^{(0)}$ возьмём последний локальный минимум частот, 
а в качестве $\mu_2^{(0)}$~--- первый
(т.\,к.~$\mathbf{E}\sbr{B(t) - B(t - \Delta t)} = 0$, 
а $\mathbf{E}\sbr{B(t_c) + K_{t_c}(t_c) - B(t - \Delta t)} = m_c > 0$).

В качестве $\sigma_j^{(0)}$ возьмём $\frac{1}{3}(\mu_1^{(0)} - \mu_2^{(0)}), 
\quad j=1, 2$.

\section{Результаты работы}

%\appendix
%\section{Исходный код}
%\label{appendix:sources}

%\usestyle{default}

%\subsection{Title}
%\label{appendix:sources:sources-name}
%\includecode[python -O linenos=1]{data/source.py}

\pagebreak

\bibliographystyle{unsrt}
\bibliography{references}

\end{document}
